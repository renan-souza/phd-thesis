%%% MY COMMANDS

\newcommand{\thesistitle}{Supporting User Steering in Large-scale Workflows with Provenance Data}
\newcommand{\thesistitlept}{Apoio a Workflows de Larga Escala Conduzidos por Usuário com Dados de Proveniência}




% Table Definitions
\newcolumntype{M}[1]{>{\centering\arraybackslash}m{#1}}
\newcommand{\mc}[2]{\multicolumn{#1}{c}{#2}}



% Colors
\definecolor{TableHeaderColor}{gray}{0.95}
\definecolor{x_darkred}{RGB}{177, 0, 0}






\newcommand{\needref}[1][?]{\colorbox{lightgray}{[R #1]}}
% Usage: \needref{} or \needref[any hint on the refs]{}

\newcommand{\eg}{\emph{e.g.},}
\newcommand{\egUpper}{\emph{E.g.},}
\newcommand{\ie}{\emph{i.e.},}
\newcommand{\etal}{\textit{et al.}}

\newcommand{\codebackground}[1]{\colorbox{black!5}{\parbox{\dimexpr\linewidth-2\fboxsep}{\fontfamily{pcr}\scriptsize#1}}}


%\newcommand{\codefont}[1]{{\fontfamily{pcr}{\small{#1}}}}
\newcommand{\codefont}[1]{\texttt{#1}}
% \texttt{printf}
% \verb$printf$ 

% Counter
\newcounter{qcounter}
\newcommand{\createQ}[2]{
    \refstepcounter{qcounter} \label{#1} {Q\ref{#1}} {#2}
}
%\newcommand{\refQ}[1]{ Q\ref{#1} }
\newcommand{\refQ}[1]{Q\ref{#1}}
%\newcommand{\refQ}[1]{#1}
% Usage: first create the reference using \createQ{label_name}, then you can make a reference to it using refQ{label_name}

\newcommand{\queries}{\hyperref[tab:queries]{Q1--Q6}}
\newcommand{\textonto}[1]{\texttt{\small{#1}}}
\newcommand{\textontohead}[1]{\textbf{\emph{#1}}}
\newcommand{\textsoftware}[1]{\textsc{#1}}

\newcommand{\alert}[1]{\textcolor{red}{#1}}



\newcommand{\defterm}[2]{\symbl{\capitalisewords{#1}}{#2}\textit{#1}}
\newcommand{\myabbrev}[2]{\abbrev{#1}{#2}{#2 (#1)}}
\newcommand{\mychap}[2]{\chapter{\capitalisewords{#1}}\label{#2}}

\newcommand{\subsubsubsection}[1]{\vspace{5mm} \noindent \emph{#1} \vspace{5mm} }

\theoremstyle{definition}
\newtheorem{definition}{Definition}[chapter]

\newenvironment{mydef}[2] % #1 => def label, def caption
{

  \begin{definition}{\textbf{#2}.}  \label {#1}
}
{
\end{definition}
}


\newcommand{\redx}{\large\textcolor{x_darkred}{\textbf{X}}}


%%%%%%%%%% TODO Notes %%%%%%%%%%%
% Usage:
% \todocmd[<optional> inline]{Text to remember}
% %\usepackage[disable]{todonotes}



\newcommandx{\revise}[2][1=]{\todo[linecolor=red,backgroundcolor=red!25,bordercolor=red,#1]{#2}}
\newcommandx{\crossref}[2][1=]{\todo[linecolor=violet,backgroundcolor=violet!25,bordercolor=violet,#1]{#2}}
\newcommandx{\formatting}[2][1=]{\todo[linecolor=green,backgroundcolor=green!25,bordercolor=green,#1]{#2}}
\newcommandx{\requestrevision}[2][1=]{\todo[linecolor=blue,backgroundcolor=blue!25,bordercolor=blue,#1]{#2}}
\newcommandx{\addintro}[2][1=]{\todo[linecolor=yellow,backgroundcolor=yellow!25,bordercolor=yellow,#1]{#2}}



\lstdefinestyle{mystyle}{
    %backgroundcolor=\color{backcolour},   
    %commentstyle=\color{codegreen},
    %keywordstyle=\color{magenta},
    numberstyle=\tiny\color{darkgray},
    stringstyle=\color{purple},
    basicstyle=\footnotesize,
    %keywordstyle=\bfseries,
    breakatwhitespace=false,         
    breaklines=true,                 
    captionpos=b,                    
    keepspaces=true,                 
    numbers=left,                    
    numbersep=5pt,                  
    showspaces=false,                
    showstringspaces=false,
    showtabs=false,                  
    tabsize=1,
    upquote=true,
    frame=single,
    %linewidth=14cm,
    morekeywords={include, printf}
}
\lstset{style=mystyle}