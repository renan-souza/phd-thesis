\chapter*{Agradecimentos}

Agradeço:

A Deus, porque até aqui nos ajudou o Senhor. 

À Gisele, minha esposa e motivação maior, por estar ao meu lado desde o primeiro dia da graduação até o último dia do doutorado. Seu amor, apoio e compreensão têm sido essencial para alcançar meus objetivos.

À minha mãe por todos sacrifícios para prover a melhor educação que pôde. Seus conselhos e orações foram especialmente importantes nesses últimos meses. 

Agradeço ao Marcos, ao meu pai, à Deuseni, a toda minha família e à família da Gisele pela força e compreensão.

À minha orientadora, Professora Marta Mattoso, 
que me orientou em todos os momentos, persistentemente, desde o início do mestrado até aqui. Com ela, muito além da pesquisa, aprendi o significado de ensino. Agradeço pela paciência, esforço e cuidado que teve comigo durante todos esses difíceis anos. 

Ao meu coorientador, Professor Patrick Valduriez, pela orientação objetiva e precisa. Agradeço-o também pelo apoio com equipamentos (Grid5000) usados nos experimentos e pelo suporte financeiro enquanto estive no período sanduíche na França.

Aos pesquisadores e colaboradores que estão ou estiveram no Laboratório da IBM Research no Brasil. Dentre os quais, alguns tornaram-se amigos próximos que participaram do meu dia-a-dia na labuta dividida entre o trabalho e o doutorado, compartilhando seus conhecimentos e palavras de ânimo. Por medo de cometer a injustiça de não mencionar alguém, ficam aqui meus sinceros agradecimentos a todos os amigos. Agradeço também aos gerentes do BRL que sempre incentivaram meu doutorado, me orientaram e permitiram minha liberação para realização das atividades acadêmicas quando necessário. Agradeço especialmente ao meu gerente Marco Netto que, além disso, aceitou fazer parte do exame de qualificação e da banca desta tese.

Ao Jonas Dias pelas valiosas sugestões dadas durante o exame de qualificação. Aos Professores Vanessa Braganholo Murta e Alexandre Lima pelos conselhos no exame de qualificação e por terem aceitado fazer parte da banca. Agradeço também à Professora Lúcia Drummond por ter aceitado compor a banca.

Aos amigos Vítor Silva e José Camata pela colaboração na pesquisa e ajuda para alcançar os resultados da tese. Aos participantes do projeto SciDISC, especialmente ao Professor Alvaro Coutinho pela orientação, incentivo e parceria nos artigos. 

Aos amigos do laboratório LIRMM/Inria em Montpellier.

À equipe administrativa do PESC e do NACAD por toda a ajuda na burocracia. À  equipe  do  NACAD também agradeço pelo  suporte  com  os  equipamentos  usados  nos  experimentos (Lobo Carneiro). Aos colaboradores do \href{http://coppetex.sourceforge.net}{COPPE-\LaTeX},   por fornecerem o fonte do template usado para escrever esta tese. Os fontes deste documento encontram-se em \href{https://github.com/renan-souza/phd-thesis}{https://github.com/renan-souza/phd-thesis}.

Finalmente, agradeço a todos meus professores que participaram da minha educação, desde a época do colégio até o doutorado. Certamente tive diversas influências positivas que me inspiraram e ajudaram a chegar até aqui.

A todos, muito, muito obrigado!!! :)



