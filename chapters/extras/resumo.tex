\begin{abstract}

\textit{Workflows} em Ciência e Engenharia Computacional (CSE) são de larga escala, requerem execução em ambientes de Processamento de Alto Desempenho (PAD) e têm a natureza exploratória da ciência.
% , \ie{} existe um enorme espaço de solução para ser explorado, de modo a inspecionar como a variação de parâmetros afeta Quantidades de Interesse, como precisão e convergência.
Durante a execução, que costuma demorar horas ou dias, usuários precisam conduzir o \textit{workflow} analisando-o e adaptando-o dinamicamente para melhorar a qualidade dos resultados ou reduzir o tempo de execução.
Entretanto, eles fazem diversas ações de condução do \textit{workflow}, que precisam ser rastreadas. Caso contrário, eles têm dificuldade de entender como e o que precisa ser conduzido, podem conduzir de forma errada, pode ser difícil explicar resultados que foram consequências das ações e pode ser impossível de reproduzir os resultados.
Para resolver esse problema, esta tese propõe uma abordagem que define os conceitos fundamentais para ações de usuário para condução de workflows; introduz a noção de proveniência de ações de condução; e um diagrama de dados compatível com o padrão W3C PROV para modelar dados de ações de condução com proveniência.
Além disso, a abordagem apresenta princípios de projeto de sistemas para gerência de dados de ações de condução através da captura, relacionamento com o restante dos dados do \textit{workflow} e armazenamento eficiente.
Duas instâncias dessa abordagem foram projetadas e implementadas: uma para ser adicionada a \textit{scripts}  paralelos e a outra é usada em um Sistema Paralelo de Gerência de \textit{Workflows}, as quais são as duas formas típicas de se executar experimentos de CSE em PAD.
Através de experimentos com casos reais da indústria de Óleo e Gás, mostra-se que a abordagem permite que usuários entendam como suas ações afetam diretamente os resultados em tempo de execução e também que os princípios de projeto foram essenciais para adicionar sobrecarga desprezível à execução dos \textit{workflows} em PAD.




\end{abstract}

