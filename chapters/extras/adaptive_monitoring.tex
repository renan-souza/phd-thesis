\section{Adaptive Monitoring Concepts}
\label{sec_adaptive_monitoring_concepts}

In this section, we present an adaptive monitoring approach that
combines monitoring and adaptation. 
It helps users following the
evolution of the workflow data being generated during execution, interesting parameters, QoIs, and result data. Since what
users find interesting may change over time, this approach allows the
user to adapt the monitoring definitions, such as which data should be
monitored and how.
The adaptive monitoring relies on online queries to
the continuously populated workflow database. Users can set up
queries (such as the ones in Tables \ref{tab:queries1} and \ref{tab:queries2}) to monitor the data, analyze monitoring results, and
adapt monitoring settings.

Monitoring works as follows. There is a query set $QS$ composed of
monitoring queries $q_i$, 
$0 \leq i \leq |QS|$,
each one to be
executed at each $\Delta t_i > 0$ time intervals. Users do not need to
specify queries at the beginning of the execution, since they do not know
everything they want to monitor. This is why $QS$ starts empty.

After users gain insights from the data, after interactive 
data analyses, they can add monitoring queries to $QS$ in an
\textit{ad-hoc} manner. 
Each $\Delta t_i$ can be adapted, meaning that users
have control of the time frame of each $q_i$ during
execution. The monitoring queries and settings are stored in the
workflow database.

Each $q_i$ execution generates a monitoring query result set $qr_{it}$, $t = \{k\Delta t_i | k \in \mathbb{N}_{\geq 0}\}$, at each
time interval $\Delta t_i$. This result set is also stored in the
workflow database.
The users have the flexibility to adapt the monitoring during
workflow execution. To do so, at each time instant $t$ after each
monitoring query result $qr_{it}$ has been generated,
the values for $\Delta t_i$ and $q_i$ are reloaded from the
workflow database. If any change has happened, it will be considered in the
next iteration $t + \Delta t_i$.
Moreover, at each certain time during
execution (also configured by the user), there is a check to verify if the user
has added new monitoring queries in $QS$. This approach takes 
advantage of the data stored online in the workflow database to enable users
to adapt monitoring settings, including which data will be monitored and
how.

